\documentclass[12pt]{article}
\usepackage{graphicx}
\usepackage{wrapfig}
\usepackage{subcaption}
\usepackage{float}
\usepackage{amsmath}
\usepackage{commath}
\usepackage{siunitx}
\usepackage{titlesec}
\usepackage{hyperref}
\usepackage{xcolor}
\usepackage{hyperref}
\usepackage[labelfont=bf]{caption}
\usepackage[font=small]{caption}


\setlength{\textwidth}{6.5in}
\setlength{\textheight}{9.9in}
\setlength{\oddsidemargin}{0.0in}
\setlength{\evensidemargin}{0.0in}
\setlength{\topmargin}{-2.5 cm}
\setlength{\parskip}{0.7\baselineskip} % space between paragraphs
\setlength{\parindent}{0 cm} % increase this if you like paragraphs to be indented

\pagestyle{empty}

\titlespacing*{\section}
  {0pt}    % left indent
  {2ex plus 1ex minus .2ex}  % space before
  {1ex plus .2ex}            % space after

\titlespacing*{\subsection}
  {0pt}{1.5ex plus .5ex minus .2ex}{0.8ex plus .2ex}

\newcommand{\e}{\mathrm{e}}
\definecolor{darkGreen}{RGB}{0, 90, 30}
\definecolor{darkPurple}{RGB}{58, 0, 119}

\begin{document}
\begin{center}
    \bf{\LARGE PHY180 Pendulum Project \\ Lab 1: Period vs. Angle and Q factor} \\
    \vspace{2 ex}
    {\large William Hu}
\end{center}

\vspace{-5ex}

\section{Introduction}
The objective of this project was to determine the acccuracy of predictions by the \textbf{damped harmonic oscillator} model of a pendulum (Wilson, 2025a). The experiments examined how a pendulum's initial amplitude $\theta_0$ affects its period $T$ and measured the $Q$ factor, which represents how slowly the amplitude $\theta$ decays per $T$. Analysis of the collected data and its uncertainties yielded several results. The pendulum was confirmed to be symmetric ($T$ for $\theta > 0$ and $\theta < 0$ are equal). $T$ increases quadratically with $\theta_0$, but after considering uncertainty, it is experimentally consistent with a constant small-angle period $T_0$ when $\abs{\theta_0}\leq0.25 \pm 3$. Using a starting amplitude within the small-angle range, $Q$ was calculated from the experimentally determined exponential decay constant $\tau$ and period $T$. The amplitude $\theta$ fits the decaying exponential closely. Considering error, the calculated $Q$ agreed with the $Q$ found by counting the number of $T$ for $\theta$ to decay by a carefully selected factor. 

Once the pendulum was built, most data collection used the Open Source Physics (OSP) Tracker (Brown, 2025) to search the recorded footage. The data, reformatted through Google Sheets, was run through Python script customized from the one provided (Wilson, 2025b) to generate best fit data and graphs.

The following equations may be referenced throughout the report:
\begin{align}
    \theta(t)&=\theta_0 ~ \e^{-t/\tau} \cos{\left(\frac{2\pi}{T}t+\phi_0 \right)} 
        \label{amplitude vs. time}\\
    \theta_{max}(t)&=\theta_0 ~ \e^{-t/\tau} \label{max amplitude vs. time}\\
    T(\theta_0) &= T_0(1 + B\theta_0 + C\theta_0^2) \label{power series}\\
    Q&=\pi\frac{\tau}{T} \label{q calculated}\\ 
    Q&=nK \label{q counted}
\end{align}

The damped harmonic oscillator model suggested that $\theta$ could be described by the function of time Equation~\ref{amplitude vs. time} (Wilson, 2025a) where Equation \ref{max amplitude vs. time} describes the maximum amplitude per oscillation. The model also predicted that $T$ did not depend on $\theta_0$ at all, that the best fit 3-term power series Equation \ref{power series} (Wilson, 2025a) would yield $B,~ C=0$ and $T=T_0$. 
Equation \ref{q calculated} (Wilson, 2025a) quantifies $Q$ in terms of other obtainable variables. By Equation~\ref{q counted}, $Q$ can also be found by counting $K$, the number of oscillations for $\theta$ to decay by a factor of $\e^{-\pi/n}$ for any $n$.

When analyzing uncertainties, two values $a \pm u(a),~b \pm u(b)$ (where $a<b$) are in fact considered experimentally indistinguishable if their error bars overlap ($a + u(a) \geq b - u(b))$. In the special case where $u(a)=u(b)$, $a$ and $b$ agree within error if $\abs{b-a}\leq 2u(a).$

\pagebreak
\section{Methods and Procedures}
{\color{darkGreen}
The string was tied around a screw, which was inserted into the mass (a tennis ball). The polypropylene string had negligible strain caused by the weight of the mass. Two stacks of books stabilized two boards which jutted off the table and clamped the string, fixing the pivot. A tripod supported the camera (a phone) from a distance, capturing slow-motion footage.
}
{\color{darkPurple}
$m_{\text{ball}}\approx\SI{60}{g}$, $m_{\text{screw}}\approx\SI{1}{g}$, and the string has a linear mass density $\lambda\approx\SI{3}{g/m}$ ($L<\SI{0.5}{m}$ for all experiments). Because of the high $m_{\text{ball}}$ to $m_{\text{string}} + m_{\text{screw}}$ ratio, the pendulum's centre of mass can be treated as the ball's centre of mass.
}

\subsection{Measuring Period vs. Initial Amplitude}
{\color{darkGreen}
The period was measured using frame-by-frame analysis of the footage in OSP Tracker, with 3 oscillations per trial to reduce type B uncertainty (after confirming that multiple oscillations did not significantly impact $T$, i.e. $Q\gg1$). A protractor taped to the end of the boards measured initial amplitudes ($\pm \pi/2$). When dropping the mass, it was ensured from another angle that the pendulum would swing on a plane perpendicular to the protractor. The string was clamped close to the protractor to prevent parallax in readings (see Figure~\ref{fig:setup1}).}
{\color{darkPurple}The length $L=58.0\pm 0.5\,\text{cm}$ was used for all trials.}

\begin{figure}[h]
    \centering
    \begin{minipage}[t]{0.30\textwidth}
        \centering
        \includegraphics[height=0.2\textheight]{input-files/setup1.jpg}
        \caption{Setup with protractor for Period vs. Initial Amplitude Experiment}
        \label{fig:setup1}
    \end{minipage}
    \hfill
    \begin{minipage}[t]{0.30\textwidth}
        \centering
        \includegraphics[height=0.2\textheight]{input-files/setup2.jpg}
        \caption{Setup with two strings for Amplitude vs. Time (Q Factor Experiment)}
        \label{fig:setup2}
    \end{minipage}
    \hfill
    \begin{minipage}[t]{0.30\textwidth}
        \centering
        \includegraphics[height=0.2\textheight]{input-files/tracker.png}
        \caption{Manually Tracking $\theta$ with OSP Tracker's protractor}
        \label{fig:tracker}
    \end{minipage}
\end{figure}
\vspace{-0.5cm}

\subsection{Measuring Q Factor}
{\color{darkGreen}
Footage was collected starting at $\theta_{max} \approx 0.25$ (within the small-angle approximation). As the pendulum could drift off plane with many oscillations, the mass was now hung from two strings of equal length to prevent motion normal to the desired plane (see Figure~\ref{fig:setup2}). The lens of the phone camera was normal to said plane and level with the pivot to reduce parallax error, since max amplitude was now measured from the footage using OSP Tracker's built-in protractor (see Figure~\ref{fig:tracker}). Not every oscillation was manually tracked. It would be unnecessary, since $\theta_{max}$ does not change meaningfully after small amounts of oscillations (it was later discovered that decay constant $\tau \gg T$). $T_0$ and $T_{final}$ were also tracked for $Q$ calculations.} 
{\color{darkPurple} The length $L=29.1 \pm 0.5\,\text{cm}$ was used.}

% TODO: rewrite
{\color{darkPurple}
$Q=nK$ (Equation~\ref{q counted}), so $u(Q)=nu(K)$. Selecting $n$ can be difficult because smaller $n$ can decrease $u(Q)$ but also result in a wider range of acceptable $K$ values (oscillations for which $\theta_{max}$ and $\theta_{final}$ agree within error), increasing $u(K)$. Data from both $n=2$ and $n=3$ were collected.
}

\subsection{Measuring Period vs. Pendulum Length}
The camera setup was improved as described in Figure \ref{fig:white}. 
$L_{\text{pendulum}} = y_{\text{pivot}} - y_{\text{ball centre}} 
= y_{\text{pivot}} - y_{\text{ball bottom}}- r_{\text{ball}}$. 
Measuring the length of the string would not be helpful as the two-string pendulum was still in use, so a metre stick (Figure \ref{fig:metre stick setup}) was used to measure $y_{\text{pivot}}$ and $y_{\text{ball bottom}}$ after changes of length. The tape was necessary, as changing viewing angle to confirm proper alignment while the stick often shifted its position. 
$y_{\text{ball bottom}}$ was measured  with eye level held as closely to the metre stick and as horizontal as achievable to reduce parallex error as shown in Figure \ref{fig:ball bottom}.

The pendulum was dropped from slightly higher than the small angle $\theta_0<0.25$ so it would stabilize after any potential human error (e.g. non-zero force) by the time it reached $\theta_0$.
$T$ was measured (in OSP Tracker) by dividing the time for 10 oscillations by 10. With the small angle approximation, the change in period is negligible, so averaging 10 oscillations reduces type B uncertainty per oscillation. 

\begin{figure}[h]
    \centering
    \begin{minipage}[t]{0.30\textwidth}
        \centering
        \includegraphics[height=0.2\textheight]{input-files/white.png}
        \caption{A higher contrast background and camera flash improved the string's visibility.}
        \label{fig:white}
    \end{minipage}
    \hfill
    \begin{minipage}[t]{0.30\textwidth}
        \centering
        \includegraphics[height=0.2\textheight]{input-files/metre stick setup.jpg}
        \caption{A metre stick taped to the wooden boards and floor.}
        \label{fig:metre stick setup}
    \end{minipage}
    \hfill
    \begin{minipage}[t]{0.30\textwidth}
        \centering
        \includegraphics[height=0.2\textheight]{input-files/ball bottom.jpg}
        \caption{View of the metre stick at the bottom of the ball.}
        \label{fig:ball bottom}
    \end{minipage}
\end{figure}
\vspace{-0.5cm}

\subsection{Measuring Q Factor vs. Length}
The same footage (multiple minutes per video) for varying length was analyzed in OSP Tracker. Max amplitudes were manually tracked. To find the decay constants and their uncertainties $\tau,\,u(\tau)$, $\theta_{\text{max}}$ vs. $t$ data for each length was fitted to Equation \ref{max amplitude vs. time} with a \href{https://github.com/planT-444/phy180-pendulum/blob/master/yippee2.py}{Python script}, which also fitted the calculated $Q$ vs. $L$ data to various curves (see Appendix \ref{trying curves}).


\section{Results and Data Analysis}
\begin{figure}[H]
    \centering
    \includegraphics[height=0.28\textheight]{output-files/period vs amplitude.png}
    \caption{
        {\color{darkPurple}
        Graph of $T$ vs. $\theta_0$ at release fitted to Equation~\ref{power series}, the 3-term power series 
        \mbox{$T(\theta_0) = T_0(1 + B\theta_0 + C\theta_0^2)$} (Equation \ref{power series}), 
        yielding $T_0=1.528 \pm \SI{0.002}{s},\:B=-0.0006 \pm \SI{0.0005}{s},\:C=0.0563 \pm \SI{0.0006}{s}.$
        $T(0)=T_0$, visually located at the vertex.
        $\theta_0$ error bars are too small to be visible.
        The large outlier at $\theta_0=0.873$ was a result of a misdrop but was not artificially discluded.
        Not all $T$ error bars hit the best fit curve, which is expected.
        }
    }
    \label{fig:period vs amplitude}
\end{figure}
\vspace{-0.5cm}
{\color{darkGreen}
The type B uncertainty in initial amplitude $u(\theta_0)=\SI{0.5}{\degree}=\SI{0.009}{rad}$ comes from the precision of the protractor's markings and the thickness of the string (non-negligible for a small protractor). 
}
{\color{darkPurple}
The error in period $u(T)=\max(u_A(T), u_B(T))$ considers both uncertainties, type A (standard error of the mean of at least 3 repeated trials $\frac{\sigma_T}{\sqrt{N}}$, ranged from 0.001 to 0.007 excluding the misdrop outlier)} 
{\color{darkGreen}
and type B (footage limitation $1/\SI{120}{FPS}/\SI{3}{oscillations}=\SI{0.003}{s}$).


As the model predicted, $B\approx0~(\,\abs{B - u(B)} < 2u(B))$, implying a symmetrical pendulum since $T(\theta)=T(1+C\theta_0^2)$ is an even function (visually, Figure \ref{fig:period vs amplitude} confirms this). However, contrary to model's prediction, $C \neq 0$ experimentally (considering error $C \gg 2u(C)$, $C$ cannot plausibly considered 0). Indeed, $T(\theta_0)$ grows quadratically with $\abs{\theta_0}$ according to the graph.

For small $\theta_0$, $u(T)=\SI{0.003}{s}$.
To determine what $\theta_0$ allow $T\approx T_0$ (small angle approximation), it must be determined when they agree within error ($T-T_0=T_0C\theta_0^2 \leq 2u(T)$) 
\begin{center}
    \begin{math}
        \abs{\theta_0} \leq \sqrt{\frac{u(T)}{T_0 C}}
            \pm \sqrt{
                    \max(
                        \frac{u(T_0)}{T_0},~ 
                        \frac{u(C)}{{C}}
                    )
                }
                \sqrt{\frac{u(T)}{T_0 C}}=0.25 \pm 0.03
    \end{math}
\end{center}

Therefore, the small angle approximation is valid for $\abs{\theta_0}\leq 0.25 \pm 0.03$. According to the residuals in Figure~\ref{fig:period vs amplitude}, the size of the error bars are constant in this range and $u(T)=\SI{0.003}{s}$ is indeed valid for $\abs{\theta_0}\leq 0.25~ (u_A(T) < u_B(T))$. However, when $\abs{\theta_0}>0.25 \pm 0.03$, $T$ depends on $\theta_0$, growing quadratically and non-negligibly.
}
\begin{figure}[H]
    \centering
    \includegraphics[height=0.28\textheight]{output-files/amplitude vs. time.png}
    \caption{
        {\color{darkPurple}
        Graph of the decaying $\theta_{\text{max}}$ vs. time fitted to the exponential
        $\theta_{\text{max}}(t) = \theta_0 \e ^ {t/\tau}$ (Equation \ref{max amplitude vs. time}), yielding $\theta_0= 0.271 \pm 0.002,\:\tau= 51.3 \pm \SI{0.5}{s}$. 
        $\abs{\Delta\theta_{max}}$ per oscillation decreases, which complicates identifying $K$ when many $\theta_{\text{max}}$ agree with $\theta_{\text{final}}$ within error. 
        $t$ error bars are too small to be visible.
        Fitting with Equation \ref{max amplitude vs. time} results in residuals that mostly overlap the curve within error. The decay can reasonably be modelled with Equation \ref{max amplitude vs. time}, but it appears to initially underestimate and eventually overestimate the true $\theta_{\text{max}}$.
        }
    }
    \label{fig:amplitude vs. time}
\end{figure}
\vspace{-0.5cm}
{\color{darkGreen}
The type B uncertainties are $u(\theta_{max})= \SI{0.2}{\degree} = \SI{0.003}{\radian}$ (smaller than before, since OSP Tracker's protractor radius can be enlarged so that the string thickness spans less ticks) and $u(t)= \SI{0.008}{s}$ ($1/\SI{120}{FPS})$.

$T = 1.083 \pm \SI{0.008}{s}$. $T$ is considered the average of $T_0,~T_{final}$ with an 
uncertainty half their distance. Using best fit values and also counting $K$, the 
three experimental $Q$ factors are:
\begin{alignat*}{2}
    Q_{fit}&=\pi\frac{\tau}{T} 
        \pm \max(\frac{u(\tau)}{\tau}, \frac{u(T)}{T})\,\pi\frac{\tau}{T}
            &=&149\pm 2 \\
    Q_2&=2(K_2 \pm u(K_2))=2(77 \pm 5)&=&154 \pm 10 \\
    Q_3&=3(K_3 \pm u(K_3))=3(55 \pm 3)&=&165 \pm 9
\end{alignat*}
$Q_2$ experimentally agrees fully with $Q_{fit}$ ($\,\abs{Q_{fit}-Q_2} \leq u(Q_2))$, while the minimum distance between $Q_3$ and $Q_{fit}$'s error bars is $\frac{5}{2}u(Q_{fit})=\frac{5}{9}u(Q_3)$, which is very close to fully agreeing (a 4\% difference). Thus, it can be concluded that the two methods for measuring $Q$ agree.
}
% RN
\begin{figure}[H]
    \centering
    \includegraphics[height=0.28\textheight]{output-files/period vs length.png}
    \caption{
        Graph of $T$ vs. $L$ fitted to $T=kL^n$, yielding $k=1.997\pm \SI{0.003}{s/m^{0.5}}$, 
        $n=0.500 \pm 0.001$.
    }
    \label{fig:period vs length}
\end{figure}
\vspace{-0.5cm}

\begin{figure}[h]
    \begin{minipage}[t]{0.28\textwidth}
        \centering
        \includegraphics[height=0.28\textheight]{output-files/period vs length.png}
        \caption{
            Graph of $T$ vs. $L$ fitted to $T=kL^n$, yielding $k=1.997\pm \SI{0.003}{s/m^{0.5}}$, 
            $n=0.500 \pm 0.001$. 
        }
        \label{fig:period vs length}
    \end{minipage}
    \hfill
    \begin{minipage}[t]{0.28\textwidth}
        \centering
        \includegraphics[height=0.28\textheight]{output-files/period vs length.png}
        \caption{
            Graph of $Q$ vs. $L$ fitted to the equation $Q=A \sqrt{L} + B$, yielding $A=400 \pm \SI{20}{m^{-0.5}}$, $B=-67 \pm 7$. For $L< \SI{3}{cm},\:Q<0$, which is impossible. However, $r_{\text{ball}}\approx \SI{3}{cm}$, so $L<\SI{3}{cm}$ represents cases with no string, and thus, no pendulum.
        }
        \label{fig:Q vs length}
    \end{minipage}
\end{figure}

\begin{figure}[H]
    
\end{figure}
\vspace{-0.5cm}

\section{Conclusion}
By analyzing experimental uncertainty, it is concluded that the period $T$ can be approximated as the small-angle period $T_0=1.528 \pm \SI{0.002}{s}$ for $\abs{\theta_0}\leq 0.25\pm0.03$ but grows quadratically for $\abs{\theta_0} > 0.25 \pm 0.03$. Note that $\abs{\theta_0}$ is used as that the pendulum was confirmed to be symmetric ($B=0$). The pendulum's $\theta_{max}$ per oscillation can be accurately modelled with exponential decay $\theta_{max}(t)=\theta_0\e ^ {\tau/t}$. $Q_{fit}=149\pm2$ calculated from $\tau$ and $T$ agrees with the $Q_2=150\pm10$ and $Q_3=165\pm9$ obtained by counting $K$ for $n=2$, $n=3$. 

For future experiments, the two-string setup which restricts non-planar motion should always be used. Measuring $T$ in other experiments will be quite error-tolerant since slight differences in $\theta_0$ negligibly change it and because $Q$ is sufficiently large for $T$ to change negligibly after a few oscillations.

\appendix
\section{Trying to fit various curves to $Q$ vs. $L$ data} \label{trying curves}

\section{Use of AI for this report}
Generative AI (ChatGPT) was only queried for basic questions about matplotlib, for help fixing \LaTeX{} compilation errors, and for general conventions in lab report formatting. No content was conceived, formulated, articulated, or revised by ChatGPT.

\pagebreak
\section{References}
    \hangindent=2em
    \noindent Brown, Douglas. \emph{Tracker Video Analysis and Modeling Tool},  
        Open Source Physics, Version 6.3.2, 2025.
        \url{https://physlets.org/tracker/}

    Wilson, Brian. \emph{PHY180 Pendulum Project (2025)}.
        Downloaded from Quercus, 2025a.

    Wilson, Brian. \emph{fit\_black\_box.py}.
        Downloaded from Quercus, 2025b.


\end{document}


