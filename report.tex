\documentclass[12pt]{article}
\usepackage{graphicx}
\usepackage{float}
%\usepackage{wrapfig}
\setlength{\textwidth}{6.5in}
\setlength{\textheight}{9.9in}
\setlength{\oddsidemargin}{0.0in}
\setlength{\evensidemargin}{0.0in}
\setlength{\topmargin}{-2.5 cm}
\setlength{\parskip}{0.7\baselineskip} % space between paragraphs
\setlength{\parindent}{0 cm} % increase this if you like paragraphs to be indented

\pagestyle{empty}

\newcommand{\e}{\mathrm{e}}

\begin{document}
\begin{center}
\bf{\LARGE 
PHY180 Pendulum Project \\ Lab 1: Period vs. Angle and Q factor 
}
\end{center}

\section{Introduction}

This document describes, briefly, how you should format a paper. The Appendix has some notes on how to use TeX. Most physicists use TeX or one of its variants (like LaTeX) to write papers because it does an excellent job of typesetting equations.

Your introduction should start with a brief summary of your entire report. What are your 1 to 4 most important results? Put them here. Once the reader has read the first paragraph, they should already know whether they want to read the rest of your paper. I recommend writing this summary last, after the conclusion. You really want the first paragraph of your introduction to be very similar to your conclusion.

The rest of your introduction should include an overview of what you did at a high level (you built a pendulum, but don't include details like what materials you used here) and include any theory that is relevant. The introduction is the primary location for introducing new equations. It's not great style to introduce new physics equations in any of the subsequent sections.

Before you write any paper ever, you need to understand who your target audience is. For this paper, pretend your target audience is a high school physics student. Obviously, your real target audience is the grader, but they will be grading your paper as if they were a high school physics student. This is important as it sets what assumptions you can make about what is common knowledge and what should be assumed to be new to the reader.

\section{Methods and Procedures}
The string was tied around a screw, which was inserted into the mass (tennis ball).
Two wooden boards clamped the string near their ends, using friction to fix the pendulum's centre of rotation. 
The boards jutted off the edge of a table, allowing the pendulum to swing freely.
A tripod supported the camera (a phone) from a distance, capturing slow-motion footage.

\subsection{Angle vs. Period}
The period was measured using slow-motion footage (120 FPS) and 3 oscillations per trial to 
reduce type B uncertainty. A protractor taped to the end of the boards measured initial amplitudes.
When dropping the mass, it was also ensured from the side that the pendulum 

This is where you discuss any important details about what you did and how you did it, including justifications for any choices you made. Did you use nylon fishing line for the pendulum because it doesn't stretch much? Great, describe that briefly here. Anything that you did specifically to reduce your uncertainties is very valuable information to include here! 

The goal is that if the target audience reads {\bf only this paper} and nothing else (especially not the instructions file you got) they should nonetheless be able to reproduce your experiment and get similar results. Your target audience (high school physics students) presumably knows how to measure lengths and times, so don't include detailed notes on how to use a ruler. That is irrelevant. But if you did something to prevent your pendulum from oscillating in elliptical (3d) motion, rather than planar (2D) motion, that would be useful to document.

You can, and in some cases should, allude to future results in this section if they affect choices you made. For example, if you decide to always release your pendulum from an angle of 10 degrees after doing the Period vs. Angle part of the experiment, describe that here even though you have not presented those results yet.

\section{Results and Data Analysis}

Sometimes this is two different sections. However, this experiment has enough parts that I would combine them. This part might want subsections for each of the 4 topics (basically each graph could have its own subsection). Depending on how clear you write, subsections may not be necessary.

This section is about presenting your raw data, interpreting it properly for the reader, and discussing your uncertainties. You will want to graph your data (no tables of data please!) with the line of best fit. You will want to explicitly describe the line of best fit (probably by referencing one of the equations from your introduction) and any other important features of the graph (slope, intercepts, whatever is important). Make concrete what the data and graph should tell the reader. All the important results that you included in your introduction and conclusion need to be explicitly explained here. Your analysis should reference every graph. For example: ``In Figure \ref{fig:FallingRock} we see that the position of a falling rock in the asteroid is well-fit by a quadratic function. This indicates that the acceleration due to gravity is approximately a constant.''

\begin{figure}[h!]
\begin{centering}
\includegraphics[width=0.6 \textwidth]{graph.png}
\label{fig:FallingRock}
\caption{Time series data of the height of a rock falling near the surface of an asteroid. Best fit line is a quadratic corresponding to an acceleration of $1.4 \pm 0.2~ \mathrm{m/s^2}$. Note how the title is not simply `Height-Time graph', which is information already present. If your errorbars are too small to be seen, specificaly state that in this caption. You should aim for your captions to explain a lot. Ideally, if a reader only reads your first paragraph of your introduction plus the captions of your graph they should understand about half of your paper!}
\end{centering}
\end{figure}

You will want to comment on how good the line of fit is, based on your uncertainties. And you will want to describe as many of the uncertainty sources as you found, include your estimated values for each uncertainty source, and then clearly state which was the largest and most important source.
It is likely that your first experiment will have a detailed analysis of uncertainties, and that the subsequent experiments will have basically the same uncertainty budgets. In this case you should not repeat the uncertainty analysis, but you should repeat the results. Something like, `As with the previous experiments, the largest source of uncertainty was due to blah and had a magnitude of blah.'

Your results section {\bf must have quantitative conclusions}. Did the two ways of measuring the Q factor agree with each other? State this clearly with a quantitative assessment. You will need to explain your criteria for assessing this. For example, if you decided that if the values overlap by less than two uncertainty ranges then you will claim they agree, you should make this criteria explicit.



\section{Conclusion}

Repeat all your judgment calls. Is period independent of angle? Does the amplitude decay exponentially, and if so do the two Q factor measurements agree? Is $T=2\sqrt{L}$ correct? Quantify how correct it is; for example, if you found the value was $1.95 \pm 0.03$ instead of 2, include that here and comment on whether your value agrees with 2; do not force the reader to make decisions, do it for them. What dependence did you find for the Q factor in terms of pendulum length? These must be clearly restated, preferably quantitatively.

You should also discuss what the largest uncertainty sources were, and mention how they could be decreased `in the future'. 

\appendix

\section{Notes on TeX}

This document is given to PHY180 students who wish to write their reports using TeX or LaTeX. You'll need to download
an appropriate TeX compiler, and optionally a front-end (or editor). I use TeXworks, but there are better GUIs for beginners. I'm old enough that I had to learn TeX before there were any nice editors.

{\bf Here's how you make something bold.} \emph{Here's a way to make something in italics.}

You can even make {\large words large} {\small or small}.

Here's a table from the document describing the pendulum project, though I added vertical lines.
\begin{table}[h]
\begin{center}
	\begin{tabular}{| c | r | c | c |}
		\hline
		Section                    & Due Date    & Weight 1 & Weight 2\\ \hline
		Lab 1: Period vs. Angle and Q Factor  & 30 Sep 2022 & 6\%  & 3\%  \\
		Lab 2: Period vs. Length and Q Factor vs. Length & 28 Oct 2022 & 6\%  & 3\%  \\
		Final Report & 2 Dec 2022 & 10\% & 16\% \\ \hline
	\end{tabular}
\caption{\label{TableName} Here's where you put the caption.}
\end{center}
\end{table}

\vspace{-0.5cm}
It usually puts a lot of space under tables. If you want less, you can use the \textbackslash vspace\{\} command. Negative numbers remove space, positive add space. Similarly, you can use \textbackslash hspace\{\} if you want horizontal space. This is rarely needed.

Note that the \% sign makes things a comment. If you want to actually type the \% sign, you need to precede it with a
\textbackslash symbol like \textbackslash \%.

Here's an equation. Making nice equations is the single biggest reason to use TeX.
\begin{equation} 
	\label{DampedHarmonic}
	\theta(t) = \theta_0 ~ \e^{-t/\tau} \cos{\left(2\pi \frac{t}{T}+\phi_0 \right)}
\end{equation}
The label command lets you refer to it like this: Equation \ref{DampedHarmonic}.
Note that you can also put equations in a paragraph with dollar signs $y(x)=mx+b$. You cannot reference them though.
Here are the other equations from the document describing the pendulum project:
$T = T_0(1  + B \theta_0 + C \theta_0^2 + \ldots )$ \\
$Q = \pi \frac{\tau}{T}$ \\
$T =k\, L^n$ \\
$T = 2 \sqrt{L}$ \\
Note that the \textbackslash\textbackslash ~ command forces a new line. The $\sim$ symbol forces a space 
which can be handy.

Here's a numbered list of things I hate
\begin{enumerate}
\item Redundancy
		\item Misalignment (TeX fixes it though)
\item Irony
\item Numbered lists
\end{enumerate}
Let's repeat this with bullet points
\begin{itemize}
\item Redundancy
		\item Misalignment 
\item Irony
\item Numbered lists
\end{itemize}

\subsection{Subsections}

Do you need subsections? Easy. You could even use \textbackslash subsubsection\{\} if you really want.

\subsection*{Subsections Without Numbers}

If you don't want numbering for subsections (or sections), use an asterisk like \textbackslash subsection*\{\}.

\subsection*{Including Graphics}

Want to include a picture? Remove the \textbackslash begin\{verbatim\} and \textbackslash end\{verbatim\} commands
below. Note, you won't see these commands in the PDF version. Verbatim changes the font and then writes everything while ignoring any formatting commands other than \textbackslash end\{verbatim\}.
\begin{verbatim}
\begin{centre}
\includegraphics[width=0.35\textwidth]{filename.pdf}
\end{centre}
\end{verbatim}


\end{document}


